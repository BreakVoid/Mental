\documentclass[11pt,a4paper]{article}
\usepackage{ctex}
\usepackage[margin=1in]{geometry}

\title{Mental特性和优化汇总}
\author{柯嵩宇}
\begin{document}

\maketitle
\tableofcontents

\newpage
\section{visitor模式:from AST to linear IR}
	以visitor模式完全手写了AST到IR的visitor。(ANTLR生成CST,用ANTLR的CST生成我的AST,然后用我写的visitor生成IR)
\section{逻辑表达式短路求值}
\subsection{naive的短路求值}
	对于二元运算的逻辑与或,可以先计算左边的元素,如果左侧表达式的结果可以决定整个表达式的值,那么就保存结果同时跳过右侧表达式的计算。
\subsection{修改逻辑与或运算的结合性}
	在superloop中,出现了一个巨大的逻辑与运算,如果按照上面的方法短路求值,那么在执行过程中会出现大量的跳转。一个比较机智的优化就是修改逻辑与或运算的结合性

\end{document}
